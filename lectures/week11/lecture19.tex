% Options for packages loaded elsewhere
\PassOptionsToPackage{unicode}{hyperref}
\PassOptionsToPackage{hyphens}{url}
%
\documentclass[
]{article}
\usepackage{amsmath,amssymb}
\usepackage{lmodern}
\usepackage{iftex}
\ifPDFTeX
  \usepackage[T1]{fontenc}
  \usepackage[utf8]{inputenc}
  \usepackage{textcomp} % provide euro and other symbols
\else % if luatex or xetex
  \usepackage{unicode-math}
  \defaultfontfeatures{Scale=MatchLowercase}
  \defaultfontfeatures[\rmfamily]{Ligatures=TeX,Scale=1}
\fi
% Use upquote if available, for straight quotes in verbatim environments
\IfFileExists{upquote.sty}{\usepackage{upquote}}{}
\IfFileExists{microtype.sty}{% use microtype if available
  \usepackage[]{microtype}
  \UseMicrotypeSet[protrusion]{basicmath} % disable protrusion for tt fonts
}{}
\makeatletter
\@ifundefined{KOMAClassName}{% if non-KOMA class
  \IfFileExists{parskip.sty}{%
    \usepackage{parskip}
  }{% else
    \setlength{\parindent}{0pt}
    \setlength{\parskip}{6pt plus 2pt minus 1pt}}
}{% if KOMA class
  \KOMAoptions{parskip=half}}
\makeatother
\usepackage{xcolor}
\IfFileExists{xurl.sty}{\usepackage{xurl}}{} % add URL line breaks if available
\IfFileExists{bookmark.sty}{\usepackage{bookmark}}{\usepackage{hyperref}}
\hypersetup{
  pdftitle={Lecture 19- Partial Pooling},
  pdfauthor={Peter Shaffery},
  hidelinks,
  pdfcreator={LaTeX via pandoc}}
\urlstyle{same} % disable monospaced font for URLs
\usepackage[margin=1in]{geometry}
\usepackage{color}
\usepackage{fancyvrb}
\newcommand{\VerbBar}{|}
\newcommand{\VERB}{\Verb[commandchars=\\\{\}]}
\DefineVerbatimEnvironment{Highlighting}{Verbatim}{commandchars=\\\{\}}
% Add ',fontsize=\small' for more characters per line
\usepackage{framed}
\definecolor{shadecolor}{RGB}{248,248,248}
\newenvironment{Shaded}{\begin{snugshade}}{\end{snugshade}}
\newcommand{\AlertTok}[1]{\textcolor[rgb]{0.94,0.16,0.16}{#1}}
\newcommand{\AnnotationTok}[1]{\textcolor[rgb]{0.56,0.35,0.01}{\textbf{\textit{#1}}}}
\newcommand{\AttributeTok}[1]{\textcolor[rgb]{0.77,0.63,0.00}{#1}}
\newcommand{\BaseNTok}[1]{\textcolor[rgb]{0.00,0.00,0.81}{#1}}
\newcommand{\BuiltInTok}[1]{#1}
\newcommand{\CharTok}[1]{\textcolor[rgb]{0.31,0.60,0.02}{#1}}
\newcommand{\CommentTok}[1]{\textcolor[rgb]{0.56,0.35,0.01}{\textit{#1}}}
\newcommand{\CommentVarTok}[1]{\textcolor[rgb]{0.56,0.35,0.01}{\textbf{\textit{#1}}}}
\newcommand{\ConstantTok}[1]{\textcolor[rgb]{0.00,0.00,0.00}{#1}}
\newcommand{\ControlFlowTok}[1]{\textcolor[rgb]{0.13,0.29,0.53}{\textbf{#1}}}
\newcommand{\DataTypeTok}[1]{\textcolor[rgb]{0.13,0.29,0.53}{#1}}
\newcommand{\DecValTok}[1]{\textcolor[rgb]{0.00,0.00,0.81}{#1}}
\newcommand{\DocumentationTok}[1]{\textcolor[rgb]{0.56,0.35,0.01}{\textbf{\textit{#1}}}}
\newcommand{\ErrorTok}[1]{\textcolor[rgb]{0.64,0.00,0.00}{\textbf{#1}}}
\newcommand{\ExtensionTok}[1]{#1}
\newcommand{\FloatTok}[1]{\textcolor[rgb]{0.00,0.00,0.81}{#1}}
\newcommand{\FunctionTok}[1]{\textcolor[rgb]{0.00,0.00,0.00}{#1}}
\newcommand{\ImportTok}[1]{#1}
\newcommand{\InformationTok}[1]{\textcolor[rgb]{0.56,0.35,0.01}{\textbf{\textit{#1}}}}
\newcommand{\KeywordTok}[1]{\textcolor[rgb]{0.13,0.29,0.53}{\textbf{#1}}}
\newcommand{\NormalTok}[1]{#1}
\newcommand{\OperatorTok}[1]{\textcolor[rgb]{0.81,0.36,0.00}{\textbf{#1}}}
\newcommand{\OtherTok}[1]{\textcolor[rgb]{0.56,0.35,0.01}{#1}}
\newcommand{\PreprocessorTok}[1]{\textcolor[rgb]{0.56,0.35,0.01}{\textit{#1}}}
\newcommand{\RegionMarkerTok}[1]{#1}
\newcommand{\SpecialCharTok}[1]{\textcolor[rgb]{0.00,0.00,0.00}{#1}}
\newcommand{\SpecialStringTok}[1]{\textcolor[rgb]{0.31,0.60,0.02}{#1}}
\newcommand{\StringTok}[1]{\textcolor[rgb]{0.31,0.60,0.02}{#1}}
\newcommand{\VariableTok}[1]{\textcolor[rgb]{0.00,0.00,0.00}{#1}}
\newcommand{\VerbatimStringTok}[1]{\textcolor[rgb]{0.31,0.60,0.02}{#1}}
\newcommand{\WarningTok}[1]{\textcolor[rgb]{0.56,0.35,0.01}{\textbf{\textit{#1}}}}
\usepackage{graphicx}
\makeatletter
\def\maxwidth{\ifdim\Gin@nat@width>\linewidth\linewidth\else\Gin@nat@width\fi}
\def\maxheight{\ifdim\Gin@nat@height>\textheight\textheight\else\Gin@nat@height\fi}
\makeatother
% Scale images if necessary, so that they will not overflow the page
% margins by default, and it is still possible to overwrite the defaults
% using explicit options in \includegraphics[width, height, ...]{}
\setkeys{Gin}{width=\maxwidth,height=\maxheight,keepaspectratio}
% Set default figure placement to htbp
\makeatletter
\def\fps@figure{htbp}
\makeatother
\setlength{\emergencystretch}{3em} % prevent overfull lines
\providecommand{\tightlist}{%
  \setlength{\itemsep}{0pt}\setlength{\parskip}{0pt}}
\setcounter{secnumdepth}{-\maxdimen} % remove section numbering
\ifLuaTeX
  \usepackage{selnolig}  % disable illegal ligatures
\fi

\title{Lecture 19- Partial Pooling}
\author{Peter Shaffery}
\date{3/30/2021}

\begin{document}
\maketitle

\hypertarget{example---radon}{%
\section{Example - Radon}\label{example---radon}}

Let's say that you would like to buy a house in Minnesota. A big concern
when buying a new house is checking its radon levels. Radon is a
colorless, odorless gas produced as part of the radioactive decay
process of Uranium. Prolonged exposure to radon can have serious (and
negative) health consequences, so it's important to buy a house without
radon!

Fortunately you've got a dataset containing household level radon
measurements all over Minnesota. Included in this dataset is the
household's county, the floor on which the measurement was made on, as
well as some other potentially useful variable. You'd like to use this
information to find a house with low radon levels.

\begin{Shaded}
\begin{Highlighting}[]
\FunctionTok{library}\NormalTok{(tidyverse)}
\FunctionTok{library}\NormalTok{(magrittr)}

\NormalTok{dat }\OtherTok{=} \FunctionTok{read.csv}\NormalTok{(}\StringTok{\textquotesingle{}../../data/radon\_mn.csv\textquotesingle{}}\NormalTok{)}
\NormalTok{dat }\SpecialCharTok{\%\textgreater{}\%}\NormalTok{ head}
\end{Highlighting}
\end{Shaded}

\begin{verbatim}
##   idnum state state2 stfips   zip region typebldg floor room basement windoor
## 1  5081    MN     MN     27 55735      5        1     1    3        N      NA
## 2  5082    MN     MN     27 55748      5        1     0    4        Y      NA
## 3  5083    MN     MN     27 55748      5        1     0    4        Y      NA
## 4  5084    MN     MN     27 56469      5        1     0    4        Y      NA
## 5  5085    MN     MN     27 55011      3        1     0    4        Y      NA
## 6  5086    MN     MN     27 55014      3        1     0    4        Y      NA
##   rep stratum wave starttm stoptm startdt stopdt activity pcterr     adjwt
## 1   2       4   41     930    930   12088  12288      2.2    9.7 1146.4992
## 2   5       2   40    1615   1615   11888  12088      2.2   14.5  471.3662
## 3   3       2   42    1030   1515   20288  21188      2.9    9.6  433.3167
## 4   2       2   24    1410   1410  122987 123187      1.0   24.3  461.6237
## 5   3       2   40     600    600   12888  13088      3.1   13.8  433.3167
## 6   5       2   22     900   1300  120287 121187      2.5   12.8  471.3662
##   dupflag zipflag cntyfips county
## 1       1       0        1 AITKIN
## 2       0       0        1 AITKIN
## 3       0       0        1 AITKIN
## 4       0       0        1 AITKIN
## 5       0       0        3  ANOKA
## 6       0       0        3  ANOKA
\end{verbatim}

\begin{Shaded}
\begin{Highlighting}[]
\FunctionTok{hist}\NormalTok{(dat}\SpecialCharTok{$}\NormalTok{activity)}
\end{Highlighting}
\end{Shaded}

\includegraphics{lecture19_files/figure-latex/unnamed-chunk-1-1.pdf}

\begin{Shaded}
\begin{Highlighting}[]
\NormalTok{dat}\SpecialCharTok{$}\NormalTok{log\_radon }\OtherTok{=} \FunctionTok{log}\NormalTok{(dat}\SpecialCharTok{$}\NormalTok{activity}\FloatTok{+.001}\NormalTok{)  }\CommentTok{\# since activity bounded convert to log scale, offset slightly to avoid issues with log(0)}
\FunctionTok{hist}\NormalTok{(dat}\SpecialCharTok{$}\NormalTok{log\_radon)}
\end{Highlighting}
\end{Shaded}

\includegraphics{lecture19_files/figure-latex/unnamed-chunk-1-2.pdf}

\hypertarget{model-1--county-level-averages}{%
\subsection{Model 1- County-level
Averages}\label{model-1--county-level-averages}}

Let's start by imagining the simplest possible model of radon levels in
the state of Minnesota:

\begin{Shaded}
\begin{Highlighting}[]
\NormalTok{pool }\OtherTok{=} \FunctionTok{mean}\NormalTok{(dat}\SpecialCharTok{$}\NormalTok{log\_radon,}\AttributeTok{na.rm=}\ConstantTok{TRUE}\NormalTok{)}
\NormalTok{pool}
\end{Highlighting}
\end{Shaded}

\begin{verbatim}
## [1] 1.208567
\end{verbatim}

It may not seem like much, but even a sample average is (technically) a
type of model: \[
y_i = \mu + \epsilon_i
\] We'll call this model our \emph{pooled} model, since it ignores any
differences in radon measurements due to county, floor of measurement,
etc. It's just \emph{pooling} all of the datapoints together and taking
a mean

Now, obviously this model has problems. For one it has \(R^2=0\) (since
the SSR term is just equal to SSE). If we're trying to estimate the
radon in house within a given county, our sample mean is probably going
to be off by quite a lot!

A simple way to account for differences by mean counties would be to
simply group our data at the county level, and then take an average:

\begin{Shaded}
\begin{Highlighting}[]
\NormalTok{sigma.y }\OtherTok{=} \FunctionTok{sd}\NormalTok{(dat}\SpecialCharTok{$}\NormalTok{log\_radon)}
\NormalTok{unpool }\OtherTok{=}\NormalTok{ dat }\SpecialCharTok{\%\textgreater{}\%} 
  \FunctionTok{group\_by}\NormalTok{(county) }\SpecialCharTok{\%\textgreater{}\%}
  \FunctionTok{summarize}\NormalTok{(}\AttributeTok{radon\_mn=}\FunctionTok{mean}\NormalTok{(log\_radon), }\AttributeTok{radon\_se=}\NormalTok{sigma.y}\SpecialCharTok{/}\FunctionTok{sqrt}\NormalTok{(}\FunctionTok{n}\NormalTok{()), }\AttributeTok{size=}\FunctionTok{n}\NormalTok{() )}
\end{Highlighting}
\end{Shaded}

Note that the county average \(\mu_j\) has standard error
\(\frac{\hat \sigma_j}{\sqrt{n}}\) (why?).

We'll call this our \emph{unpooled} model, since it is \textbf{not}
ignoring differences due to county. We are not treating the data as one
homogeneous pool anymore, but rather separating out the data into
groups.

There are two ways we could think about this model. The first is a model
which estimates a separate mean \(\mu_j\) for each county \(j\):

\[
y_i = \mu_{j[i]} + \epsilon_i 
\] Here I am introducing a special index notation \(j[i]\), which means
that the county index \(j\) which corresponds to datapoint \(i\).

We could have also written this model in our usual regression format but
introducing dummy variables for each county. Recall that, because we
have more than two counties, we must \emph{one-hot encode} the
``county'' variable, that is create a binary indicator variable \(C_j\),
for each of the \(j=1,...,K\) counties. That model would look like:

\[
y_i = \beta_0 + \sum_{j=1}^K \beta_j C_{ji} + \epsilon_i 
\] Where \(\beta_0 + \beta_j = \mu_j\)

While this model does a good job accounting for variation in radon
levels between counties, it has a problem:

\begin{Shaded}
\begin{Highlighting}[]
\NormalTok{dat}\SpecialCharTok{$}\NormalTok{county }\SpecialCharTok{\%\textgreater{}\%}\NormalTok{ table }\SpecialCharTok{\%\textgreater{}\%}\NormalTok{ sort}
\end{Highlighting}
\end{Shaded}

\begin{verbatim}
## .
##       MAHNOMEN         MURRAY         WILKIN           COOK       FILLMORE 
##              1              1              1              2              2 
##    LACQUIPARLE      MILLELACS           POPE           ROCK        STEVENS 
##              2              2              2              2              2 
## YELLOWMEDICINE         BECKER       BIGSTONE          DODGE         ISANTI 
##              2              3              3              3              3 
##      KANDIYOHI        KITTSON        LINCOLN         NOBLES         NORMAN 
##              3              3              3              3              3 
##     PENNINGTON       RENVILLE           TODD       WATONWAN         AITKIN 
##              3              3              3              3              4 
##         BENTON          BROWN       CHIPPEWA     CLEARWATER     COTTONWOOD 
##              4              4              4              4              4 
##        KANABEC LAKEOFTHEWOODS         MEEKER       NICOLLET      PIPESTONE 
##              4              4              4              4              4 
##           POLK         SIBLEY          SWIFT       TRAVERSE         WASECA 
##              4              4              4              4              4 
##           CASS        HUBBARD        JACKSON        LESUEUR        REDWOOD 
##              5              5              5              5              5 
##         WADENA         CARVER        CHISAGO      FARIBAULT        HOUSTON 
##              5              6              6              6              6 
##           PINE        WABASHA       BELTRAMI    KOOCHICHING         MARTIN 
##              6              6              7              7              7 
##      SHERBURNE           LYON      OTTERTAIL        DOUGLAS       FREEBORN 
##              7              8              8              9              9 
##           LAKE       MARSHALL       MORRISON        CARLTON           CLAY 
##              9              9              9             10             10 
##         ITASCA           RICE          SCOTT         STEELE       CROWWING 
##             10             10             10             10             11 
##          MOWER      BLUEEARTH         MCLEOD         WINONA         WRIGHT 
##             12             13             13             13             13 
##        GOODHUE         ROSEAU        OLMSTED        STEARNS         RAMSEY 
##             14             14             23             24             32 
##     WASHINGTON          ANOKA         DAKOTA       HENNEPIN        STLOUIS 
##             42             47             58             99            113
\end{verbatim}

A lot of counties only have a handful of measurements! What this means
is that, for these counties, our estimates of \(\mu_j\) are going to
have high standard error (solid line shows the pooled mean):

\begin{Shaded}
\begin{Highlighting}[]
\FunctionTok{ggplot}\NormalTok{(unpool,}\FunctionTok{aes}\NormalTok{(}\AttributeTok{x=}\NormalTok{size,}\AttributeTok{y=}\NormalTok{radon\_mn,}\AttributeTok{ymin=}\NormalTok{radon\_mn}\SpecialCharTok{+}\NormalTok{radon\_se,}\AttributeTok{ymax=}\NormalTok{radon\_mn}\SpecialCharTok{{-}}\NormalTok{radon\_se)) }\SpecialCharTok{+}
  \FunctionTok{geom\_point}\NormalTok{() }\SpecialCharTok{+}
  \FunctionTok{geom\_errorbar}\NormalTok{() }\SpecialCharTok{+}
  \FunctionTok{geom\_hline}\NormalTok{(}\AttributeTok{yintercept=}\NormalTok{pool) }\SpecialCharTok{+}
  \FunctionTok{labs}\NormalTok{(}\AttributeTok{x=}\StringTok{\textquotesingle{}Sample Size\textquotesingle{}}\NormalTok{, }\AttributeTok{y=}\StringTok{\textquotesingle{}Estimated mu\textquotesingle{}}\NormalTok{)}
\end{Highlighting}
\end{Shaded}

\includegraphics{lecture19_files/figure-latex/unnamed-chunk-5-1.pdf} So
on the one hand, we had the pooled model. This had the advantage of
having a very large sample size, so its estimate was highly
\emph{precise}, but by ignoring county-level variation it has low
\emph{accuracy}. On the other hand, we have the unpooled model which
could accurately account for county-level variation, but for small
counties its estimates weren't precise.

What we need is a compromise model between these two approaches, a
\textbf{partial pooling} model. It turns out that the model which will
accomplish this compromise looks like this:

\[
\begin{split}
y_i &= \mu_{j[i]} + \epsilon_i\\
\mu_j &= \alpha + \eta_j\\
\end{split}
\] Where \(\eta_j \sim N(0, \sigma^2_{\mu})\) (and similarly
\(\epsilon_i \sim N(0, \sigma^2)\)). We will refer to the \(\eta_j\) as
\textbf{group-level errors}, in contrast with the \(\epsilon_i\) which
are \textbf{individual-level errors}.

Let's take a look at the estimated \(\mu_j\) and their associated
standard errors. Don't worry about the code right now, what matters is
the output plot:

\begin{Shaded}
\begin{Highlighting}[]
\FunctionTok{library}\NormalTok{(lme4)}
\FunctionTok{library}\NormalTok{(arm)}

\NormalTok{partial.pool }\OtherTok{=} \FunctionTok{lmer}\NormalTok{(log\_radon}\SpecialCharTok{\textasciitilde{}}\NormalTok{(}\DecValTok{1}\SpecialCharTok{|}\NormalTok{county), }\AttributeTok{data=}\NormalTok{dat)}
\NormalTok{partial.pool }\OtherTok{=} \FunctionTok{data.frame}\NormalTok{(}\FunctionTok{ranef}\NormalTok{(partial.pool)}\SpecialCharTok{$}\NormalTok{county}\SpecialCharTok{+}\FunctionTok{fixef}\NormalTok{(partial.pool),}\FunctionTok{se.ranef}\NormalTok{(partial.pool)}\SpecialCharTok{$}\NormalTok{county)}
\NormalTok{partial.pool}\SpecialCharTok{$}\NormalTok{county }\OtherTok{=} \FunctionTok{rownames}\NormalTok{(partial.pool)}
\NormalTok{partial.pool}\SpecialCharTok{$}\NormalTok{size }\OtherTok{=}\NormalTok{ unpool}\SpecialCharTok{$}\NormalTok{size}
\FunctionTok{names}\NormalTok{(partial.pool) }\OtherTok{=} \FunctionTok{c}\NormalTok{(}\StringTok{\textquotesingle{}radon\_mn\textquotesingle{}}\NormalTok{,}\StringTok{\textquotesingle{}radon\_se\textquotesingle{}}\NormalTok{,}\StringTok{\textquotesingle{}county\textquotesingle{}}\NormalTok{,}\StringTok{\textquotesingle{}size\textquotesingle{}}\NormalTok{)}

\NormalTok{plt.df }\OtherTok{=} \FunctionTok{rbind}\NormalTok{(unpool,partial.pool)}
\NormalTok{plt.df}\SpecialCharTok{$}\NormalTok{model }\OtherTok{=} \FunctionTok{rep}\NormalTok{(}\FunctionTok{c}\NormalTok{(}\StringTok{\textquotesingle{}No Pooling\textquotesingle{}}\NormalTok{,}\StringTok{\textquotesingle{}Partial Pooling\textquotesingle{}}\NormalTok{),}\AttributeTok{each=}\FunctionTok{nrow}\NormalTok{(unpool))}
\NormalTok{plt.df }\SpecialCharTok{\%\textless{}\textgreater{}\%} \FunctionTok{filter}\NormalTok{(size}\SpecialCharTok{\textless{}}\DecValTok{30}\NormalTok{)}
\NormalTok{plt.df}\SpecialCharTok{$}\NormalTok{size }\OtherTok{=}\NormalTok{ plt.df}\SpecialCharTok{$}\NormalTok{size }\SpecialCharTok{+} \FunctionTok{runif}\NormalTok{(}\FunctionTok{nrow}\NormalTok{(plt.df),}\SpecialCharTok{{-}}\DecValTok{5}\NormalTok{,}\DecValTok{5}\NormalTok{)}

\FunctionTok{ggplot}\NormalTok{(plt.df,}\FunctionTok{aes}\NormalTok{(}\AttributeTok{x=}\NormalTok{size,}\AttributeTok{y=}\NormalTok{radon\_mn,}\AttributeTok{ymin=}\NormalTok{radon\_mn}\SpecialCharTok{+}\NormalTok{radon\_se,}\AttributeTok{ymax=}\NormalTok{radon\_mn}\SpecialCharTok{{-}}\NormalTok{radon\_se)) }\SpecialCharTok{+}
  \FunctionTok{geom\_point}\NormalTok{() }\SpecialCharTok{+}
  \FunctionTok{geom\_errorbar}\NormalTok{() }\SpecialCharTok{+}
  \FunctionTok{geom\_hline}\NormalTok{(}\AttributeTok{yintercept=}\NormalTok{pool) }\SpecialCharTok{+}
  \FunctionTok{facet\_wrap}\NormalTok{(}\SpecialCharTok{\textasciitilde{}}\NormalTok{model)}
\end{Highlighting}
\end{Shaded}

\includegraphics{lecture19_files/figure-latex/unnamed-chunk-6-1.pdf}

As expected, the partial-pooling estimate has compromised between the
pooled and unpooled estimates of the \(\mu\). We see that in the partial
pooling all the \(\mu_j\) have been pulled closer to the pooled mean,
and that their standard errors have been substantially equalized.

But why does this work? What about introducing that second level of
randomness into our model causes the estimates to push together?

There are two (equivalent) explanations. The first is simply an
approximation of \$\hat{\mu}\_j\^{}\{(\text{partial pooled})\} \$: \[
\hat \mu_j^{(\text{partial pooled})} \approx \frac{ \frac{n_j}{\sigma^2_y} \bar y_j + \frac{1}{\sigma^2_{\mu}} \bar y  }{\frac{n_j}{\sigma^2_y} + \frac{1}{\sigma^2_{\mu}}}
\] Here we clearly see that the partial-pooling estimate is a weighted
average between the pooled and unpooled estimates. When either \(n_j\)
(the number of observations in county \(j\)) or \(\sigma^2_{\mu}\) is
large then the unpooled estimate dominates the average. On the other
hand, for the counties with small sample sizes, or when \(\sigma_y^2\)
is large, then the pool estimate takes over.

While this formula makes clear \emph{what} is happening, it doesn't
provide much insight into \emph{why} it's happening.

Let's look at \(\sigma^2_{\mu}\). What is the interpretation of this
parameter? Well, by definition:

\[
\sigma^2_{\mu} = \text{Var}[(\mu_j - \alpha)^2]
\] That is, \(\sigma^2_{\mu}\) controls (on average) how far apart the
\(\mu_j\) can be from their mean, and consequently from each other. Put
another way, if we have two county \(\mu_j\) and \(\mu_k\), you can show
that: \[
E[(\mu_j - \mu_k)^2] = 2 \sigma^2_{\mu}
\] Now consider one of the counties with a very high sample size, say
St.~Louis county (n=113). This gives us a very confident estimate of
\(\mu \approx .8\) for St.~Louis' average log-radon levels. Moreover,
from our partially pooled model we have an estimate of
\(\sigma_{\mu}=.33\) (don't worry about how we got this just yet). That
means that, roughly speaking, we can expect 99\% of the other \(\mu_j\)
to fall with 3 standard deviations of St.~Louis, ie. the range
\(.8 \pm .99\). Thus, even for the counties with very few observations,
we already have a lot of information about what the plausible values of
their \(\mu_j\) could be. Even if the estimate is still a little less
precise, we're almost positive that it won't be less than
\(\approx -.2\).

Speaking a little more qualitatively, our partial pooling model allows
different county groups to \emph{share information} with each other. By
assuming a common distribution on the \(\mu_j\), with a variance
\(\sigma^2_{\eta}\) we are effectively curtailing the possible range
that an estimate of \(\mu_j\) can take.

\hypertarget{model-2---county-level-regressions}{%
\subsection{Model 2 - County-level
Regressions}\label{model-2---county-level-regressions}}

So our partially-pooled county model is all well and good, it gives us a
nice estimate for the \textbf{county-level mean log-radon levels}. But
it isn't the whole story, what about the floor variable? Radon seeps up
from the ground, so we would expect there to be a relationship between
floor of measurement and radon activity.

As with our simple model, we have three options.

The first option is a simple pooled regression:

\[
y_i = \beta_0 + \beta_1 x_i + \epsilon_i
\] Here \(x_i\) is our floor indicator (\(x_i=1\) if the measurement was
made in a basement, and is 0 otherwise).

\hypertarget{aside-what-counts-as-pooling}{%
\paragraph{Aside: What counts as
``Pooling''?}\label{aside-what-counts-as-pooling}}

\textbf{\textless aside\textgreater{}}

Depending on your perspective, you could also consider this model to be
``unpooled'' (of the previous type) where the basement indicator is the
grouping variable. While this is the case in this specific example
because of the definition of \(x_i\), everything here can also be
applied to cases where \(x_i\) is continuous. In that case it would no
longer be appropriate to think of \(x_i\) as a grouping variable, and
thus the model would no longer have the interpretation as an unpooled
model. For this reason I will be referring to this as the ``pooled''
model.

\textbf{\textless/aside\textgreater{}}

We might also perform an ``unpooled'' regression, that is a regression
which includes county-level information. Recall that the ``dummy
variable'' approach gives us two options for this.

The first is a simple regression including both the floor variable
(\(x_i\)) as well as the county-level indicators that we saw from the
last example (\(C_{ji}\)): \[
y_i = \beta_0 + \beta_1 x_i + \sum_{j=2}^K \beta_j C_{ji} + \epsilon_i 
\]

We might also want to include \emph{interaction} terms between \(x_i\)
and the \(C_{ji}\), that is add a
\(\sum_{j=K+1}^{2K} \beta_j C_{ji} x_i\) term to the above model. For
simplicity, we will not deal with the interaction model just yet, but
keep it in mind.

Now, observe that our model with the county level indicators could also
be written: \[
y_i = \beta_{0j[i]} + \beta_1 x_i + + \epsilon_i 
\] Where \(\beta_{0j} = \beta_0 + \beta_j\). That is, our indicator
variable model is equivalent to estimating separate intercept terms for
each county (this was the logic behind the Simpson's Paradox example way
back in Lecture 7).

Let's fit both the pooled and unpooled variables, and look at the
results for a few counties.

\begin{Shaded}
\begin{Highlighting}[]
\NormalTok{pooled }\OtherTok{=} \FunctionTok{lm}\NormalTok{(log\_radon }\SpecialCharTok{\textasciitilde{}}\NormalTok{ floor, }\AttributeTok{data=}\NormalTok{dat)}
\NormalTok{unpooled }\OtherTok{=}  \FunctionTok{lm}\NormalTok{(log\_radon }\SpecialCharTok{\textasciitilde{}}\NormalTok{ floor }\SpecialCharTok{+}\NormalTok{ county, }\AttributeTok{data=}\NormalTok{dat)}

\NormalTok{plt.df }\OtherTok{=}\NormalTok{ dat}
\NormalTok{plt.df}\SpecialCharTok{$}\NormalTok{pooled }\OtherTok{=} \FunctionTok{predict}\NormalTok{(pooled)}
\NormalTok{plt.df}\SpecialCharTok{$}\NormalTok{unpooled }\OtherTok{=} \FunctionTok{predict}\NormalTok{(unpooled)}
\NormalTok{counties }\OtherTok{=} \FunctionTok{c}\NormalTok{(}\StringTok{\textquotesingle{}LACQUIPARLE\textquotesingle{}}\NormalTok{,}\StringTok{\textquotesingle{}AITKIN\textquotesingle{}}\NormalTok{,}\StringTok{\textquotesingle{}KOOCHICHING\textquotesingle{}}\NormalTok{,}\StringTok{\textquotesingle{}STLOUIS\textquotesingle{}}\NormalTok{)}
\NormalTok{plt.df }\SpecialCharTok{\%\textless{}\textgreater{}\%} \FunctionTok{filter}\NormalTok{( county }\SpecialCharTok{\%in\%}\NormalTok{ counties )}

\FunctionTok{ggplot}\NormalTok{(plt.df, }\FunctionTok{aes}\NormalTok{(}\AttributeTok{x=}\NormalTok{floor)) }\SpecialCharTok{+}
  \FunctionTok{geom\_point}\NormalTok{(}\FunctionTok{aes}\NormalTok{(}\AttributeTok{y=}\NormalTok{log\_radon)) }\SpecialCharTok{+} 
  \FunctionTok{geom\_line}\NormalTok{(}\FunctionTok{aes}\NormalTok{(}\AttributeTok{y=}\NormalTok{pooled)) }\SpecialCharTok{+}
  \FunctionTok{geom\_line}\NormalTok{(}\FunctionTok{aes}\NormalTok{(}\AttributeTok{y=}\NormalTok{unpooled),}\AttributeTok{linetype=}\StringTok{\textquotesingle{}dashed\textquotesingle{}}\NormalTok{) }\SpecialCharTok{+}
  \FunctionTok{facet\_wrap}\NormalTok{(}\SpecialCharTok{\textasciitilde{}}\NormalTok{county)}
\end{Highlighting}
\end{Shaded}

\includegraphics{lecture19_files/figure-latex/unnamed-chunk-7-1.pdf}
Look at the unpooled model (dashed line) for Lac Qui Parle county. It's
predictions are \emph{way} higher than everywhere else, but there's only
two datapoints! Again, because the sample size is so small there is a
substantial amount of uncertainty in \(\beta_{0j}\) (or, equivalently,
\(\beta_j\)).

As before, we need to apply partial pooling in order to strike a balance
here: \[
\begin{split}
y_i &= \beta_{0j[i]} + \beta_1 x_i + \epsilon_i\\
\beta_{0j} &\sim N(\mu_{\beta},\sigma_{\beta}^2)\\
\end{split}
\]

This model is \emph{very} similar to the partial pooling example we have
already seen, except now we have added an additional coefficient
\(\beta_1\) which is constant across counties.

\hypertarget{aside-terminology}{%
\paragraph{Aside: Terminology}\label{aside-terminology}}

\textbf{\textless aside\textgreater{}} Models like the above are
sometimes called \textbf{mixed effects models}, that is it contains
coefficients (``effects'') which are both \textbf{fixed} (the
\(\beta_1\)) and others which are \textbf{random} (like \(\beta_{0j}\)).
Models which only contain random effects are sometimes called
\textbf{random effects models}.

In general, I'm not a big fan of the term ``random effects'', and this
course I will use either the term \textbf{partial pooling} (as above) or
\textbf{hierarchical model} (alternately \textbf{multilevel model}).
It's not that the term ``random effects'' is wrong \emph{per se}, but in
my opinion ``partial pooling'' describes more accurately how this class
of model operates. Similarly ``hierarchical model'', as we will see,
describes an alternative method of understanding how these models can be
interpreted.

\textbf{\textless{}\aside\textgreater{}}

In R, the standard function we use to fit this kind of model is
\texttt{lmer} (from the package \texttt{lme4}, \texttt{lmer}=``Linear
Mixed Effects in R'' and \texttt{lme}=``Linear Mixed Effects''). This
function operates fairly similar to the usual \texttt{lm} function,
except the functions look a little different:

\begin{Shaded}
\begin{Highlighting}[]
\NormalTok{partial }\OtherTok{=}\NormalTok{ lme4}\SpecialCharTok{::}\FunctionTok{lmer}\NormalTok{(log\_radon }\SpecialCharTok{\textasciitilde{}}\NormalTok{ floor }\SpecialCharTok{+}\NormalTok{ (}\DecValTok{1}\SpecialCharTok{|}\NormalTok{county) }\SpecialCharTok{{-}} \DecValTok{1}\NormalTok{, }\AttributeTok{data=}\NormalTok{dat)}
\end{Highlighting}
\end{Shaded}

Here note the \texttt{(1\textbar{}county)} term, which corresponds to
the \(\beta_{0j}\). Unlike the regression table of \texttt{lm},
\texttt{lmer} contains a little less information:

\begin{Shaded}
\begin{Highlighting}[]
\FunctionTok{summary}\NormalTok{(partial)}
\end{Highlighting}
\end{Shaded}

\begin{verbatim}
## Linear mixed model fit by REML ['lmerMod']
## Formula: log_radon ~ floor + (1 | county) - 1
##    Data: dat
## 
## REML criterion at convergence: 2485.7
## 
## Scaled residuals: 
##     Min      1Q  Median      3Q     Max 
## -8.2498 -0.4658  0.0791  0.5998  3.4901 
## 
## Random effects:
##  Groups   Name        Variance Std.Dev.
##  county   (Intercept) 2.2037   1.4845  
##  Residual             0.7464   0.8639  
## Number of obs: 878, groups:  county, 85
## 
## Fixed effects:
##       Estimate Std. Error t value
## floor -0.68842    0.08602  -8.003
\end{verbatim}

Notice that this doesn't include any actual estimates of the
\(\beta_{0j}\). This is typical for dealing with models of this form, as
often there will simply be too many \(\beta_{0j}\) for that output to be
really meaningful. Instead we get estimates of
\(\sigma^2_{\beta}\approx .1\) and \(\sigma^2_{\epsilon} = .74)\), as
well as \(\beta_1=-.76\). If we want the specific estimates of the
\emph{random effects} we need:

\begin{Shaded}
\begin{Highlighting}[]
\FunctionTok{ranef}\NormalTok{(partial) }\SpecialCharTok{\%\textgreater{}\%}\NormalTok{ head}
\end{Highlighting}
\end{Shaded}

\begin{verbatim}
## $county
##                (Intercept)
## AITKIN           0.7680406
## ANOKA            0.8538704
## BECKER           1.3547609
## BELTRAMI         1.4640912
## BENTON           1.3136129
## BIGSTONE         1.3597361
## BLUEEARTH        1.9912485
## BROWN            1.8196611
## CARLTON          0.9675275
## CARVER           0.7387944
## CASS             1.3124984
## CHIPPEWA         1.5954167
## CHISAGO          0.9836702
## CLAY             2.1434851
## CLEARWATER       1.2191559
## COOK             0.5690371
## COTTONWOOD       0.0918021
## CROWWING         1.1181647
## DAKOTA           1.3193306
## DODGE            1.6178280
## DOUGLAS          1.6678573
## FARIBAULT        0.5995716
## FILLMORE         1.1838081
## FREEBORN         2.0186992
## GOODHUE          1.9003475
## HENNEPIN         1.3452033
## HOUSTON          1.6686977
## HUBBARD          1.1452881
## ISANTI           0.9491919
## ITASCA           0.8443501
## JACKSON          1.8925295
## KANABEC          1.1400856
## KANDIYOHI        1.7187454
## KITTSON          1.4102957
## KOOCHICHING      0.7645549
## LACQUIPARLE      2.5167648
## LAKE             0.3849690
## LAKEOFTHEWOODS   1.7073318
## LESUEUR          1.6313534
## LINCOLN          1.9077494
## LYON             1.8835514
## MAHNOMEN         1.0168304
## MARSHALL         1.5274713
## MARTIN           0.9892880
## MCLEOD           0.9079796
## MEEKER           1.2289989
## MILLELACS        0.7431100
## MORRISON         1.1033950
## MOWER            1.6199068
## MURRAY           1.8624668
## NICOLLET         1.9961438
## NOBLES           1.7322671
## NORMAN           1.1147041
## OLMSTED          1.2840541
## OTTERTAIL        1.5410471
## PENNINGTON       0.9712845
## PINE             0.7169368
## PIPESTONE        1.7084807
## POLK             1.5728786
## POPE             1.0943394
## RAMSEY           1.1440261
## REDWOOD          1.8513911
## RENVILLE         1.4918344
## RICE             1.8118221
## ROCK             1.1112913
## ROSEAU           1.6088190
## SCOTT            1.6923213
## SHERBURNE        1.0637233
## SIBLEY           1.1457662
## STEARNS          1.4324093
## STEELE           1.5283794
## STEVENS          1.5324167
## STLOUIS          0.8574258
## SWIFT            0.9103614
## TODD             1.5394694
## TRAVERSE         1.8444304
## WABASHA          1.7269074
## WADENA           1.1925580
## WASECA           0.5604706
## WASHINGTON       1.3526439
## WATONWAN         2.4155224
## WILKIN           1.6658837
## WINONA           1.5749066
## WRIGHT           1.6013991
## YELLOWMEDICINE   1.0149470
\end{verbatim}

\begin{Shaded}
\begin{Highlighting}[]
\FunctionTok{fixef}\NormalTok{(partial)}
\end{Highlighting}
\end{Shaded}

\begin{verbatim}
##      floor 
## -0.6884206
\end{verbatim}

Let's add in the predictions from our partially-pooled regression:

\begin{Shaded}
\begin{Highlighting}[]
\NormalTok{plt.df }\OtherTok{=}\NormalTok{ dat}
\NormalTok{plt.df}\SpecialCharTok{$}\NormalTok{pooled }\OtherTok{=} \FunctionTok{predict}\NormalTok{(pooled)}
\NormalTok{plt.df}\SpecialCharTok{$}\NormalTok{unpooled }\OtherTok{=} \FunctionTok{predict}\NormalTok{(unpooled)}
\NormalTok{plt.df}\SpecialCharTok{$}\NormalTok{partial }\OtherTok{=} \FunctionTok{predict}\NormalTok{(partial)}
\NormalTok{counties }\OtherTok{=} \FunctionTok{c}\NormalTok{(}\StringTok{\textquotesingle{}LACQUIPARLE\textquotesingle{}}\NormalTok{,}\StringTok{\textquotesingle{}AITKIN\textquotesingle{}}\NormalTok{,}\StringTok{\textquotesingle{}KOOCHICHING\textquotesingle{}}\NormalTok{,}\StringTok{\textquotesingle{}STLOUIS\textquotesingle{}}\NormalTok{)}
\NormalTok{plt.df }\SpecialCharTok{\%\textless{}\textgreater{}\%} \FunctionTok{filter}\NormalTok{( county }\SpecialCharTok{\%in\%}\NormalTok{ counties )}

\FunctionTok{ggplot}\NormalTok{(plt.df, }\FunctionTok{aes}\NormalTok{(}\AttributeTok{x=}\NormalTok{floor)) }\SpecialCharTok{+}
  \FunctionTok{geom\_point}\NormalTok{(}\FunctionTok{aes}\NormalTok{(}\AttributeTok{y=}\NormalTok{log\_radon)) }\SpecialCharTok{+} 
  \FunctionTok{geom\_line}\NormalTok{(}\FunctionTok{aes}\NormalTok{(}\AttributeTok{y=}\NormalTok{pooled)) }\SpecialCharTok{+}
  \FunctionTok{geom\_line}\NormalTok{(}\FunctionTok{aes}\NormalTok{(}\AttributeTok{y=}\NormalTok{unpooled),}\AttributeTok{linetype=}\StringTok{\textquotesingle{}dashed\textquotesingle{}}\NormalTok{) }\SpecialCharTok{+}
  \FunctionTok{geom\_line}\NormalTok{(}\FunctionTok{aes}\NormalTok{(}\AttributeTok{y=}\NormalTok{partial),}\AttributeTok{linetype=}\StringTok{\textquotesingle{}dotted\textquotesingle{}}\NormalTok{,}\AttributeTok{color=}\StringTok{\textquotesingle{}red\textquotesingle{}}\NormalTok{) }\SpecialCharTok{+}
  \FunctionTok{facet\_wrap}\NormalTok{(}\SpecialCharTok{\textasciitilde{}}\NormalTok{county)}
\end{Highlighting}
\end{Shaded}

\includegraphics{lecture19_files/figure-latex/unnamed-chunk-11-1.pdf}
Just like with our first example, we see that the partial pooling
approach creates a ``compromise'' between the pooled and unpooled
models. When the county-level sample size is low, the partially pooled
prediction is very close to the pooled estimate (Aitkin and Koochiching
counties), but when the sample size is high the partially pooled model
stays near the unpooled regression. Note that the regression for Lac Qui
Parle county is still a good bit higher than any other county! Even just
given those two measurements, it would appear that Lac Qui Parle has
higher overall radon levels, but nowhere near as high as we would have
gotten with the unpooled estimate.

As with the previous example, we can see this effect in an approximation
of the partially pooled estimates:

\[
\hat \beta_{0j}^{(\text{partial pooled})} \approx \frac{ \frac{n_j}{\sigma^2_y} }{\frac{n_j}{\sigma^2_y}  + \frac{1}{\sigma^2_{\beta}}} (\bar y_j + \beta_1 \bar x_j)  + \frac{ \frac{1}{\sigma^2_{\beta}} }{\frac{n_j}{\sigma^2_y} + \frac{1}{\sigma^2_{\beta}}} \mu_{\beta}
\] Notice that \((\bar y_j + \beta_1 \bar x_j)\) is the estimate
\(\hat \beta_{0j}\) from our unpooled model.

\hypertarget{model-3---varying-slopes}{%
\subsection{Model 3 - Varying Slopes}\label{model-3---varying-slopes}}

So the final variant we will look at today is the logical extension of
the previous model. So far, the models we have seen have all been
\emph{varying intercept} models; in the context of the example they
assume that baseline radon might be different between counties, but that
the overall effect of the floor the measurement was on is consistent.

This is actually a pretty reasonable assumption for this example, even
though we might expect different regions of the state to have different
radon production, once the gas is in the air the physics probably don't
vary much between St.~Louis and Lac Qui Parle.

Nevertheless, for completeness let's look at a \emph{varying slopes}
model. Whereas the previous approach was an extension of the model: \[
y_i = \beta_0 + \beta_1 x_i + \sum_{j=2}^K \beta_j C_{ji} + \epsilon_i 
\] Now we will extend the interaction model: \[
y_i = \beta_0 + \beta_1 x_i + \sum_{j=2}^K \beta_j C_{ji} + \sum_{j=K+1}^{2K} \beta_j C_{ji} x_i + \epsilon_i 
\] Observe that this model is equivalent to: \[
y_i = \beta_{0j[i]} + \beta_{1j[i]} x_i + \epsilon_i 
\] So for each county we would be running completely separate
regressions. As before, this will produce some wild estimates when
\(n_j\) is small:

\begin{Shaded}
\begin{Highlighting}[]
\NormalTok{pooled }\OtherTok{=} \FunctionTok{lm}\NormalTok{(log\_radon }\SpecialCharTok{\textasciitilde{}}\NormalTok{ floor, }\AttributeTok{data=}\NormalTok{dat)}
\NormalTok{unpooled }\OtherTok{=}  \FunctionTok{lm}\NormalTok{(log\_radon }\SpecialCharTok{\textasciitilde{}}\NormalTok{ floor}\SpecialCharTok{*}\NormalTok{county, }\AttributeTok{data=}\NormalTok{dat)}

\NormalTok{plt.df }\OtherTok{=}\NormalTok{ dat}
\NormalTok{plt.df}\SpecialCharTok{$}\NormalTok{pooled }\OtherTok{=} \FunctionTok{predict}\NormalTok{(pooled)}
\NormalTok{plt.df}\SpecialCharTok{$}\NormalTok{unpooled }\OtherTok{=} \FunctionTok{predict}\NormalTok{(unpooled)}
\NormalTok{counties }\OtherTok{=} \FunctionTok{c}\NormalTok{(}\StringTok{\textquotesingle{}LACQUIPARLE\textquotesingle{}}\NormalTok{,}\StringTok{\textquotesingle{}AITKIN\textquotesingle{}}\NormalTok{,}\StringTok{\textquotesingle{}KOOCHICHING\textquotesingle{}}\NormalTok{,}\StringTok{\textquotesingle{}STLOUIS\textquotesingle{}}\NormalTok{)}
\NormalTok{plt.df }\SpecialCharTok{\%\textless{}\textgreater{}\%} \FunctionTok{filter}\NormalTok{( county }\SpecialCharTok{\%in\%}\NormalTok{ counties )}

\FunctionTok{ggplot}\NormalTok{(plt.df, }\FunctionTok{aes}\NormalTok{(}\AttributeTok{x=}\NormalTok{floor)) }\SpecialCharTok{+}
  \FunctionTok{geom\_point}\NormalTok{(}\FunctionTok{aes}\NormalTok{(}\AttributeTok{y=}\NormalTok{log\_radon)) }\SpecialCharTok{+} 
  \FunctionTok{geom\_line}\NormalTok{(}\FunctionTok{aes}\NormalTok{(}\AttributeTok{y=}\NormalTok{pooled)) }\SpecialCharTok{+}
  \FunctionTok{geom\_line}\NormalTok{(}\FunctionTok{aes}\NormalTok{(}\AttributeTok{y=}\NormalTok{unpooled),}\AttributeTok{linetype=}\StringTok{\textquotesingle{}dashed\textquotesingle{}}\NormalTok{) }\SpecialCharTok{+}
  \FunctionTok{facet\_wrap}\NormalTok{(}\SpecialCharTok{\textasciitilde{}}\NormalTok{county)}
\end{Highlighting}
\end{Shaded}

\includegraphics{lecture19_files/figure-latex/unnamed-chunk-12-1.pdf}

Our final partially pooled ``version'' of this model would be: \[
\begin{split}
y_i &= \beta_{0j[i]} + \beta_{1j[i]} x_i + \epsilon_i\\
\beta_{0j} &\sim N(\mu_{\beta_0},\sigma_{\beta_0}^2)\\
\beta_{1j} &\sim N(\mu_{\beta_1},\sigma_{\beta_1}^2)\\
\end{split}
\] We can also fit this type of model with \texttt{lmer}. Similar to how
\texttt{lm} interprets interaction terms \texttt{floor*county},
\texttt{lmer} will expand \texttt{(floor\textbar{}county)} to include
\texttt{(1\textbar{}county)} as well:

\begin{Shaded}
\begin{Highlighting}[]
\NormalTok{partial }\OtherTok{=} \FunctionTok{lmer}\NormalTok{(log\_radon }\SpecialCharTok{\textasciitilde{}}\NormalTok{ (floor}\SpecialCharTok{|}\NormalTok{county), }\AttributeTok{data=}\NormalTok{dat)}

\NormalTok{plt.df }\OtherTok{=}\NormalTok{ dat}
\NormalTok{plt.df}\SpecialCharTok{$}\NormalTok{pooled }\OtherTok{=} \FunctionTok{predict}\NormalTok{(pooled)}
\NormalTok{plt.df}\SpecialCharTok{$}\NormalTok{unpooled }\OtherTok{=} \FunctionTok{predict}\NormalTok{(unpooled)}
\NormalTok{plt.df}\SpecialCharTok{$}\NormalTok{partial }\OtherTok{=} \FunctionTok{predict}\NormalTok{(partial)}
\NormalTok{counties }\OtherTok{=} \FunctionTok{c}\NormalTok{(}\StringTok{\textquotesingle{}LACQUIPARLE\textquotesingle{}}\NormalTok{,}\StringTok{\textquotesingle{}AITKIN\textquotesingle{}}\NormalTok{,}\StringTok{\textquotesingle{}KOOCHICHING\textquotesingle{}}\NormalTok{,}\StringTok{\textquotesingle{}STLOUIS\textquotesingle{}}\NormalTok{)}
\NormalTok{plt.df }\SpecialCharTok{\%\textless{}\textgreater{}\%} \FunctionTok{filter}\NormalTok{( county }\SpecialCharTok{\%in\%}\NormalTok{ counties )}

\FunctionTok{ggplot}\NormalTok{(plt.df, }\FunctionTok{aes}\NormalTok{(}\AttributeTok{x=}\NormalTok{floor)) }\SpecialCharTok{+}
  \FunctionTok{geom\_point}\NormalTok{(}\FunctionTok{aes}\NormalTok{(}\AttributeTok{y=}\NormalTok{log\_radon)) }\SpecialCharTok{+} 
  \FunctionTok{geom\_line}\NormalTok{(}\FunctionTok{aes}\NormalTok{(}\AttributeTok{y=}\NormalTok{pooled)) }\SpecialCharTok{+}
  \FunctionTok{geom\_line}\NormalTok{(}\FunctionTok{aes}\NormalTok{(}\AttributeTok{y=}\NormalTok{unpooled),}\AttributeTok{linetype=}\StringTok{\textquotesingle{}dashed\textquotesingle{}}\NormalTok{) }\SpecialCharTok{+}
  \FunctionTok{geom\_line}\NormalTok{(}\FunctionTok{aes}\NormalTok{(}\AttributeTok{y=}\NormalTok{partial),}\AttributeTok{linetype=}\StringTok{\textquotesingle{}dotted\textquotesingle{}}\NormalTok{,}\AttributeTok{color=}\StringTok{\textquotesingle{}red\textquotesingle{}}\NormalTok{) }\SpecialCharTok{+}
  \FunctionTok{facet\_wrap}\NormalTok{(}\SpecialCharTok{\textasciitilde{}}\NormalTok{county)}
\end{Highlighting}
\end{Shaded}

\includegraphics{lecture19_files/figure-latex/unnamed-chunk-13-1.pdf}

\end{document}
