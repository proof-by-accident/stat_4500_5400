% Options for packages loaded elsewhere
\PassOptionsToPackage{unicode}{hyperref}
\PassOptionsToPackage{hyphens}{url}
%
\documentclass[
]{article}
\usepackage{amsmath,amssymb}
\usepackage{lmodern}
\usepackage{iftex}
\ifPDFTeX
  \usepackage[T1]{fontenc}
  \usepackage[utf8]{inputenc}
  \usepackage{textcomp} % provide euro and other symbols
\else % if luatex or xetex
  \usepackage{unicode-math}
  \defaultfontfeatures{Scale=MatchLowercase}
  \defaultfontfeatures[\rmfamily]{Ligatures=TeX,Scale=1}
\fi
% Use upquote if available, for straight quotes in verbatim environments
\IfFileExists{upquote.sty}{\usepackage{upquote}}{}
\IfFileExists{microtype.sty}{% use microtype if available
  \usepackage[]{microtype}
  \UseMicrotypeSet[protrusion]{basicmath} % disable protrusion for tt fonts
}{}
\makeatletter
\@ifundefined{KOMAClassName}{% if non-KOMA class
  \IfFileExists{parskip.sty}{%
    \usepackage{parskip}
  }{% else
    \setlength{\parindent}{0pt}
    \setlength{\parskip}{6pt plus 2pt minus 1pt}}
}{% if KOMA class
  \KOMAoptions{parskip=half}}
\makeatother
\usepackage{xcolor}
\IfFileExists{xurl.sty}{\usepackage{xurl}}{} % add URL line breaks if available
\IfFileExists{bookmark.sty}{\usepackage{bookmark}}{\usepackage{hyperref}}
\hypersetup{
  pdftitle={Lecture 20- Partial Pooling},
  pdfauthor={Peter Shaffery},
  hidelinks,
  pdfcreator={LaTeX via pandoc}}
\urlstyle{same} % disable monospaced font for URLs
\usepackage[margin=1in]{geometry}
\usepackage{color}
\usepackage{fancyvrb}
\newcommand{\VerbBar}{|}
\newcommand{\VERB}{\Verb[commandchars=\\\{\}]}
\DefineVerbatimEnvironment{Highlighting}{Verbatim}{commandchars=\\\{\}}
% Add ',fontsize=\small' for more characters per line
\usepackage{framed}
\definecolor{shadecolor}{RGB}{248,248,248}
\newenvironment{Shaded}{\begin{snugshade}}{\end{snugshade}}
\newcommand{\AlertTok}[1]{\textcolor[rgb]{0.94,0.16,0.16}{#1}}
\newcommand{\AnnotationTok}[1]{\textcolor[rgb]{0.56,0.35,0.01}{\textbf{\textit{#1}}}}
\newcommand{\AttributeTok}[1]{\textcolor[rgb]{0.77,0.63,0.00}{#1}}
\newcommand{\BaseNTok}[1]{\textcolor[rgb]{0.00,0.00,0.81}{#1}}
\newcommand{\BuiltInTok}[1]{#1}
\newcommand{\CharTok}[1]{\textcolor[rgb]{0.31,0.60,0.02}{#1}}
\newcommand{\CommentTok}[1]{\textcolor[rgb]{0.56,0.35,0.01}{\textit{#1}}}
\newcommand{\CommentVarTok}[1]{\textcolor[rgb]{0.56,0.35,0.01}{\textbf{\textit{#1}}}}
\newcommand{\ConstantTok}[1]{\textcolor[rgb]{0.00,0.00,0.00}{#1}}
\newcommand{\ControlFlowTok}[1]{\textcolor[rgb]{0.13,0.29,0.53}{\textbf{#1}}}
\newcommand{\DataTypeTok}[1]{\textcolor[rgb]{0.13,0.29,0.53}{#1}}
\newcommand{\DecValTok}[1]{\textcolor[rgb]{0.00,0.00,0.81}{#1}}
\newcommand{\DocumentationTok}[1]{\textcolor[rgb]{0.56,0.35,0.01}{\textbf{\textit{#1}}}}
\newcommand{\ErrorTok}[1]{\textcolor[rgb]{0.64,0.00,0.00}{\textbf{#1}}}
\newcommand{\ExtensionTok}[1]{#1}
\newcommand{\FloatTok}[1]{\textcolor[rgb]{0.00,0.00,0.81}{#1}}
\newcommand{\FunctionTok}[1]{\textcolor[rgb]{0.00,0.00,0.00}{#1}}
\newcommand{\ImportTok}[1]{#1}
\newcommand{\InformationTok}[1]{\textcolor[rgb]{0.56,0.35,0.01}{\textbf{\textit{#1}}}}
\newcommand{\KeywordTok}[1]{\textcolor[rgb]{0.13,0.29,0.53}{\textbf{#1}}}
\newcommand{\NormalTok}[1]{#1}
\newcommand{\OperatorTok}[1]{\textcolor[rgb]{0.81,0.36,0.00}{\textbf{#1}}}
\newcommand{\OtherTok}[1]{\textcolor[rgb]{0.56,0.35,0.01}{#1}}
\newcommand{\PreprocessorTok}[1]{\textcolor[rgb]{0.56,0.35,0.01}{\textit{#1}}}
\newcommand{\RegionMarkerTok}[1]{#1}
\newcommand{\SpecialCharTok}[1]{\textcolor[rgb]{0.00,0.00,0.00}{#1}}
\newcommand{\SpecialStringTok}[1]{\textcolor[rgb]{0.31,0.60,0.02}{#1}}
\newcommand{\StringTok}[1]{\textcolor[rgb]{0.31,0.60,0.02}{#1}}
\newcommand{\VariableTok}[1]{\textcolor[rgb]{0.00,0.00,0.00}{#1}}
\newcommand{\VerbatimStringTok}[1]{\textcolor[rgb]{0.31,0.60,0.02}{#1}}
\newcommand{\WarningTok}[1]{\textcolor[rgb]{0.56,0.35,0.01}{\textbf{\textit{#1}}}}
\usepackage{graphicx}
\makeatletter
\def\maxwidth{\ifdim\Gin@nat@width>\linewidth\linewidth\else\Gin@nat@width\fi}
\def\maxheight{\ifdim\Gin@nat@height>\textheight\textheight\else\Gin@nat@height\fi}
\makeatother
% Scale images if necessary, so that they will not overflow the page
% margins by default, and it is still possible to overwrite the defaults
% using explicit options in \includegraphics[width, height, ...]{}
\setkeys{Gin}{width=\maxwidth,height=\maxheight,keepaspectratio}
% Set default figure placement to htbp
\makeatletter
\def\fps@figure{htbp}
\makeatother
\setlength{\emergencystretch}{3em} % prevent overfull lines
\providecommand{\tightlist}{%
  \setlength{\itemsep}{0pt}\setlength{\parskip}{0pt}}
\setcounter{secnumdepth}{-\maxdimen} % remove section numbering
\ifLuaTeX
  \usepackage{selnolig}  % disable illegal ligatures
\fi

\title{Lecture 20- Partial Pooling}
\author{Peter Shaffery}
\date{3/30/2021}

\begin{document}
\maketitle

\hypertarget{example---radon}{%
\section{Example - Radon}\label{example---radon}}

Let's say that you would like to buy a house in Minnesota. A big concern
when buying a new house is checking its radon levels. Radon is a
colorless, odorless gas produced as part of the radioactive decay
process of Uranium. Prolonged exposure to radon can have serious (and
negative) health consequences, so it's important to buy a house without
radon!

Fortunately you've got a dataset containing household level radon
measurements all over Minnesota. Included in this dataset is the
household's county, the floor on which the measurement was made on, as
well as some other potentially useful variable. You'd like to use this
information to find a house with low radon levels.

\begin{Shaded}
\begin{Highlighting}[]
\FunctionTok{library}\NormalTok{(tidyverse)}
\FunctionTok{library}\NormalTok{(magrittr)}
\FunctionTok{library}\NormalTok{(lme4)}

\NormalTok{radon }\OtherTok{=} \FunctionTok{read.csv}\NormalTok{(}\StringTok{\textquotesingle{}../../data/radon\_mn.csv\textquotesingle{}}\NormalTok{)}
\NormalTok{radon}\SpecialCharTok{$}\NormalTok{log\_radon }\OtherTok{=} \FunctionTok{log}\NormalTok{(radon}\SpecialCharTok{$}\NormalTok{activity}\FloatTok{+.001}\NormalTok{)}

\NormalTok{radon }\SpecialCharTok{\%\textgreater{}\%}\NormalTok{ head}
\end{Highlighting}
\end{Shaded}

\begin{verbatim}
##   idnum state state2 stfips   zip region typebldg floor room basement windoor
## 1  5081    MN     MN     27 55735      5        1     1    3        N      NA
## 2  5082    MN     MN     27 55748      5        1     0    4        Y      NA
## 3  5083    MN     MN     27 55748      5        1     0    4        Y      NA
## 4  5084    MN     MN     27 56469      5        1     0    4        Y      NA
## 5  5085    MN     MN     27 55011      3        1     0    4        Y      NA
## 6  5086    MN     MN     27 55014      3        1     0    4        Y      NA
##   rep stratum wave starttm stoptm startdt stopdt activity pcterr     adjwt
## 1   2       4   41     930    930   12088  12288      2.2    9.7 1146.4992
## 2   5       2   40    1615   1615   11888  12088      2.2   14.5  471.3662
## 3   3       2   42    1030   1515   20288  21188      2.9    9.6  433.3167
## 4   2       2   24    1410   1410  122987 123187      1.0   24.3  461.6237
## 5   3       2   40     600    600   12888  13088      3.1   13.8  433.3167
## 6   5       2   22     900   1300  120287 121187      2.5   12.8  471.3662
##   dupflag zipflag cntyfips county    log_radon
## 1       1       0        1 AITKIN 0.7889118025
## 2       0       0        1 AITKIN 0.7889118025
## 3       0       0        1 AITKIN 1.0650555051
## 4       0       0        1 AITKIN 0.0009995003
## 5       0       0        3  ANOKA 1.1317246401
## 6       0       0        3  ANOKA 0.9166906519
\end{verbatim}

Recall from last lecture, our \emph{varying-intercepts, partial-pooling}
model:

\[
\begin{split}
y_i &= \alpha_{j[i]} + \beta_1 x_i + \epsilon_i\\
\alpha_j& = \mu_{\alpha} + \eta_j\\
\end{split}
\] By relating the \(\alpha_j\) to each other through the model
\(\alpha_j \sim N(\mu_{\alpha},\sigma^2_{\alpha})\) we allow the
different counties to \emph{share} information with each other, while
also allowing for different counties to have different baseline
log-radon.

Let's fit that model

\begin{Shaded}
\begin{Highlighting}[]
\NormalTok{partial }\OtherTok{=} \FunctionTok{lmer}\NormalTok{(log\_radon }\SpecialCharTok{\textasciitilde{}}\NormalTok{ floor }\SpecialCharTok{+}\NormalTok{ (}\DecValTok{1}\SpecialCharTok{|}\NormalTok{county), }\AttributeTok{data=}\NormalTok{radon)}
\end{Highlighting}
\end{Shaded}

Now, let's say that in addition to our dataset containing
household-level radon measurements we \emph{also} have a dataset
containing county-level data

\begin{Shaded}
\begin{Highlighting}[]
\NormalTok{county }\OtherTok{=} \FunctionTok{read.csv}\NormalTok{(}\StringTok{\textquotesingle{}../../data/county\_radon.csv\textquotesingle{}}\NormalTok{)}
\NormalTok{county }\SpecialCharTok{\%\textgreater{}\%}\NormalTok{ head}
\end{Highlighting}
\end{Shaded}

\begin{verbatim}
##   stfips ctfips st     cty     lon    lat    Uppm
## 1      1      1 AL AUTAUGA -86.643 32.534 1.78331
## 2      1      3 AL BALDWIN -87.750 30.661 1.38323
## 3      1      5 AL BARBOUR -85.393 31.870 2.10105
## 4      1      7 AL    BIBB -87.126 32.998 1.67313
## 5      1      9 AL  BLOUNT -86.568 33.981 1.88501
## 6      1     11 AL BULLOCK -85.716 32.100 2.46112
\end{verbatim}

We see that, among other things, this dataset contains a measure of the
county's level of Uraniam (the column \texttt{Uppm}). Since radon is a
product of uranium decay, we would expect a county's overall uranium
levels to be predictive of its radon baseline. Let's combine this county
data with our estimates \(\hat \alpha_j\)

\begin{Shaded}
\begin{Highlighting}[]
\NormalTok{plt.df }\OtherTok{=} \FunctionTok{ranef}\NormalTok{(partial)}\SpecialCharTok{$}\NormalTok{county}
\NormalTok{plt.df}\SpecialCharTok{$}\NormalTok{county }\OtherTok{=}\NormalTok{ plt.df }\SpecialCharTok{\%\textgreater{}\%}\NormalTok{ rownames}
\FunctionTok{names}\NormalTok{(plt.df)}\OtherTok{=}\FunctionTok{c}\NormalTok{(}\StringTok{\textquotesingle{}coef\textquotesingle{}}\NormalTok{,}\StringTok{\textquotesingle{}cty\textquotesingle{}}\NormalTok{)}

\NormalTok{plt.df }\SpecialCharTok{\%\textless{}\textgreater{}\%} \FunctionTok{left\_join}\NormalTok{(county }\SpecialCharTok{\%\textgreater{}\%} \FunctionTok{filter}\NormalTok{(st}\SpecialCharTok{==}\StringTok{\textquotesingle{}MN\textquotesingle{}}\NormalTok{),}\AttributeTok{by=}\FunctionTok{c}\NormalTok{(}\StringTok{\textquotesingle{}cty\textquotesingle{}}\NormalTok{)) }

\FunctionTok{ggplot}\NormalTok{(plt.df, }\FunctionTok{aes}\NormalTok{(}\AttributeTok{x=}\NormalTok{Uppm,}\AttributeTok{y=}\NormalTok{coef)) }\SpecialCharTok{+}
  \FunctionTok{geom\_point}\NormalTok{()}
\end{Highlighting}
\end{Shaded}

\includegraphics{lecture20_files/figure-latex/unnamed-chunk-4-1.pdf}

\begin{Shaded}
\begin{Highlighting}[]
\FunctionTok{lm}\NormalTok{(coef }\SpecialCharTok{\textasciitilde{}}\NormalTok{ Uppm, }\AttributeTok{data=}\NormalTok{plt.df) }\SpecialCharTok{\%\textgreater{}\%}\NormalTok{ summary}
\end{Highlighting}
\end{Shaded}

\begin{verbatim}
## 
## Call:
## lm(formula = coef ~ Uppm, data = plt.df)
## 
## Residuals:
##     Min      1Q  Median      3Q     Max 
## -0.5772 -0.1066  0.0089  0.1315  0.3559 
## 
## Coefficients:
##             Estimate Std. Error t value Pr(>|t|)    
## (Intercept) -0.35164    0.06785  -5.182 1.46e-06 ***
## Uppm         0.32442    0.05948   5.454 4.75e-07 ***
## ---
## Signif. codes:  0 '***' 0.001 '**' 0.01 '*' 0.05 '.' 0.1 ' ' 1
## 
## Residual standard error: 0.1942 on 85 degrees of freedom
## Multiple R-squared:  0.2593, Adjusted R-squared:  0.2506 
## F-statistic: 29.75 on 1 and 85 DF,  p-value: 4.747e-07
\end{verbatim}

Clearly county UPPM predicts \(\alpha_j\), the county's baseline radon
levels. Can we account for this relationship in our model? Yes!

\[
\begin{split}
\text{LOG_RADON}_i &= \beta_0 + \beta_1 \text{FLOOR}_i + \sum_j \beta_{j[i]} \text{COUNTY}_{j[i]} + \epsilon_i\\
\beta_j &=  \gamma_0 + \gamma_1 \text{UPPM}_j + \eta_j
\end{split}
\]

Where \(\eta_j \sim N(0, \sigma^2_{\beta})\)

Fundamentally, this is not different than our partial pooling model
\(\alpha_j \sim N(\mu_{\alpha},\sigma^2_{\alpha})\). In both cases, we
are simply specifying a linear model for the \(\beta_j\), the only
difference is in the mean of that model: \(\mu_{\beta}\) vs
\(\gamma_0 + \gamma_1 \text{UPPM}_j + \eta_j\).

How can we interpret this model? One obvious way we have already seen,
it simply relates the baseline log-radon to county-level features. But
how is that different than just including UPPM as an independent
variable in our log-radon regression?

Well, observe that we can plug the ``upper'' level of the model directly
into the ``lower'' level: \[
\begin{split}
\text{LOG_RADON}_i &= \beta_0 + \beta_1 \text{FLOOR}_i + \sum_j (\gamma_0 + \gamma_1 \text{UPPM}_{j[i]} + \eta_{j[i]}) \text{ COUNTY}_{j[i]}  + \epsilon_i\\
&= \beta_0' + \beta_1 \text{FLOOR}_i + \sum_j (\gamma_0 + \gamma_1 \text{UPPM}_{j[i]} + \eta_{j[i]}) \text{ COUNTY}_{j[i]}  + \epsilon_i\\
\end{split}
\]

Now, because \(\text{ COUNTY}_{j[i]}\) is an indicator variable, only
one term in that sum will end up in the final regression for any given
\(i\), hence we can just write this as: \[
y_i = \beta_0 + \beta_1 \text{FLOOR}_i + \gamma_0 + \gamma_1 \text{UPPM}_{j[i]} + \eta_{j[i]} + \epsilon_i
\]

Combining up like terms: \[
\begin{split}
y_i &= (\beta_0 + \gamma_0) + \beta_1 \text{FLOOR}_i + \gamma_1 \text{UPPM}_{j[i]} + (\eta_{j[i]} + \epsilon_i)\\
&= \beta_0' + \beta_1 \text{FLOOR}_i + \beta_2 \text{UPPM}_i + \epsilon_i'\\
\end{split}
\] Now, this \emph{looks} like the model that just plugs UPPM directly
into the regression, but pay close attention to the error terms: \[
\epsilon_i' = \eta_{j[i]} + \epsilon_i
\] These error terms are not independent! Two error terms from the same
county \(j\) will share a \(\eta_{j}\) factor. From this we can show a
few interesting facts. First off,
\(\text{Var}[\epsilon_i] = \sigma^2_y + \sigma^2_{\beta}\). Second, we
have that if \(i\) and \(k\) are the indices of two measurements from
the same county (ie. \(j[i]=j[k]\)) then
\(\text{Cov}[\epsilon_i,\epsilon_k] = \sigma^2_{\beta}\). Finally, only
if \(i\) and \(k\) correspond to measurements from \emph{different}
counties is \(\text{Cov}[\epsilon_i,\epsilon_k] =0\).

Finally, finally, observe that there is one more way that we could write
this (same) model: \[
\begin{split}
\text{LOG_RADON}_i &= \alpha_{j[i]} + \beta_1 \text{FLOOR}_i + \beta_2 \text{UPPM}_{j[i]} + \epsilon_i\\
\alpha_{j[i]} &=  \gamma_0 + \eta_j
\end{split}
\] And observe that we could combine these two levels of the model
together and obtain the same result as if we included UPPM in the
``upper'' level of the model instead. This final form is how
\texttt{lmer} prefers county-level data to be incorporated:

\begin{Shaded}
\begin{Highlighting}[]
\NormalTok{dat }\OtherTok{=}\NormalTok{ radon }\SpecialCharTok{\%\textgreater{}\%} \FunctionTok{left\_join}\NormalTok{(county,}\AttributeTok{by=}\FunctionTok{c}\NormalTok{(}\StringTok{\textquotesingle{}county\textquotesingle{}}\OtherTok{=}\StringTok{\textquotesingle{}cty\textquotesingle{}}\NormalTok{))}
\NormalTok{mod }\OtherTok{=} \FunctionTok{lmer}\NormalTok{(log\_radon }\SpecialCharTok{\textasciitilde{}}\NormalTok{ floor }\SpecialCharTok{+}\NormalTok{ Uppm }\SpecialCharTok{+}\NormalTok{ (}\DecValTok{1}\SpecialCharTok{|}\NormalTok{county), }\AttributeTok{data=}\NormalTok{dat)}
\FunctionTok{summary}\NormalTok{(mod)}
\end{Highlighting}
\end{Shaded}

\begin{verbatim}
## Linear mixed model fit by REML ['lmerMod']
## Formula: log_radon ~ floor + Uppm + (1 | county)
##    Data: dat
## 
## REML criterion at convergence: 8134.2
## 
## Scaled residuals: 
##     Min      1Q  Median      3Q     Max 
## -9.2876 -0.6682  0.0262  0.6319  4.0129 
## 
## Random effects:
##  Groups   Name        Variance Std.Dev.
##  county   (Intercept) 0.1693   0.4115  
##  Residual             0.5815   0.7626  
## Number of obs: 3480, groups:  county, 85
## 
## Fixed effects:
##             Estimate Std. Error t value
## (Intercept)  1.48645    0.06529  22.767
## floor       -0.93998    0.03941 -23.853
## Uppm         0.02440    0.02864   0.852
## 
## Correlation of Fixed Effects:
##       (Intr) floor 
## floor -0.128       
## Uppm  -0.562  0.002
\end{verbatim}

\hypertarget{when-is-hierarchical-modelingpartial-pooling-effective}{%
\section{When is Hierarchical Modeling/Partial Pooling
Effective?}\label{when-is-hierarchical-modelingpartial-pooling-effective}}

Typically hierarchical modeling is effective when pooling and
non-pooling are \emph{uneffective}. This sounds a little circular, but
it's the simplest heuristic. If \(\sigma^2_{\beta}\) is small relative
to \(\sigma^2_y\) then there isn't going to be much improvement over the
pooled model. Conversely if \(\sigma_{\beta}^2\) is much larger than
\(\sigma_y\) then the unpooled model will win out.

This leads us to the \emph{intraclass correlation}:
\(\rho = \frac{\sigma_{\beta}^2}{(\sigma_{\beta}^2 + \sigma_y^2)}\).
When \(\rho\) is close to 0 then the grouping contains almost no
information, and the hierarchical structure isn't adding much to the
model. On the other hand, when \(\rho\) is close to 1 then within a
grouping the \(y_i\) will be very similar (setting aside differences due
to the independent variables), and so an unpooled model may be more
appropriate.

\begin{Shaded}
\begin{Highlighting}[]
\FunctionTok{summary}\NormalTok{(partial)}
\end{Highlighting}
\end{Shaded}

\begin{verbatim}
## Linear mixed model fit by REML ['lmerMod']
## Formula: log_radon ~ floor + (1 | county)
##    Data: radon
## 
## REML criterion at convergence: 2295.4
## 
## Scaled residuals: 
##     Min      1Q  Median      3Q     Max 
## -8.4712 -0.5262  0.0333  0.5716  3.3349 
## 
## Random effects:
##  Groups   Name        Variance Std.Dev.
##  county   (Intercept) 0.1077   0.3281  
##  Residual             0.7422   0.8615  
## Number of obs: 878, groups:  county, 85
## 
## Fixed effects:
##             Estimate Std. Error t value
## (Intercept)  1.44336    0.05527   26.12
## floor       -0.76191    0.08255   -9.23
## 
## Correlation of Fixed Effects:
##       (Intr)
## floor -0.305
\end{verbatim}

From the above we have that \(\hat \sigma_y^2 = .074\) and
\(\hat \sigma_{\beta}^2=.11\), giving as an intraclass correlation of
\(\rho \approx .6\). This is indicative of both county-, and
individual-level variation being present in the model, and helps explain
why the partial pooling (ie. hiearchical) model is an effective choice.

\hypertarget{can-i-hypothesis-test-the-hierarchical-structure}{%
\subsection{Can I Hypothesis Test the Hierarchical
Structure?}\label{can-i-hypothesis-test-the-hierarchical-structure}}

Not really. Much like how it can be difficult to interpret the p-values
of a one-hot encoded categorical variable with a large number of levels,
p-values in the context of a hierachical model are challenging to make
sense of. If only a small number of county-levels produce
``significantly'' non-zero \(\hat \alpha_j\), does that mean that the
county variable is an appropriate grouping-variable? It's difficult to
say.

Moreover, significance testing kind of misses the point. Typically in a
hierarchical modeling scenario, we don't really care whether St.~Louis
has significantly different log-radon levels than Lac Qui Parle. We're
using the model hierarchy to allow our model to control for
county-county variability (or intra-group correlation) in a principled
way.

\hypertarget{what-can-i-use-instead}{%
\subsection{What Can I Use Instead?}\label{what-can-i-use-instead}}

We \emph{do} still have ANOVA for models containing only categorical
variables, deviance to compare nested models, and AIC for pretty much
any case. All of these become more complex due to the hiearchical
structure, however.

For one, it's challenging to characterize the ``number of parameters'',
since we've allowed some parameters to be random. Instead define an
``effective number of parameters'' (or ``effective degrees of
freedom''), which is an estimate of how much pooling the model performs.

Typically it's easiest to have these quantities computed using a package
or just the \texttt{AIC} function from \texttt{stats}:

\begin{Shaded}
\begin{Highlighting}[]
\NormalTok{pool }\OtherTok{=} \FunctionTok{lm}\NormalTok{(log\_radon}\SpecialCharTok{\textasciitilde{}}\NormalTok{floor, }\AttributeTok{data=}\NormalTok{radon)}
\NormalTok{unpool }\OtherTok{=} \FunctionTok{lm}\NormalTok{(log\_radon}\SpecialCharTok{\textasciitilde{}}\NormalTok{floor}\SpecialCharTok{+}\NormalTok{county, }\AttributeTok{data=}\NormalTok{radon)}
\NormalTok{interact }\OtherTok{=} \FunctionTok{lm}\NormalTok{(log\_radon}\SpecialCharTok{\textasciitilde{}}\NormalTok{floor}\SpecialCharTok{*}\NormalTok{county, }\AttributeTok{data=}\NormalTok{radon)}
\NormalTok{re }\OtherTok{=} \FunctionTok{lmer}\NormalTok{(log\_radon}\SpecialCharTok{\textasciitilde{}}\NormalTok{floor }\SpecialCharTok{+}\NormalTok{ (}\DecValTok{1}\SpecialCharTok{|}\NormalTok{county), }\AttributeTok{data=}\NormalTok{radon)}

\FunctionTok{AIC}\NormalTok{(pool,unpool,interact,re) }\SpecialCharTok{\%\textgreater{}\%} \FunctionTok{arrange}\NormalTok{(AIC)}
\end{Highlighting}
\end{Shaded}

\begin{verbatim}
##           df      AIC
## re         4 2303.399
## unpool    87 2316.784
## interact 145 2340.451
## pool       3 2348.828
\end{verbatim}

\hypertarget{hiearchical-generalized-linear-models}{%
\section{Hiearchical, Generalized, Linear
Models}\label{hiearchical-generalized-linear-models}}

We can pretty straightforwardly extend the hiearchical modeling approach
to the world of GLMs. Say that instead of working with log-radon, we
converted the radon activity measurement to a ``counts'' scale and used
a poisson regression instead:

\begin{Shaded}
\begin{Highlighting}[]
\NormalTok{radon}\SpecialCharTok{$}\NormalTok{activity\_count }\OtherTok{=} \DecValTok{10}\SpecialCharTok{*}\NormalTok{radon}\SpecialCharTok{$}\NormalTok{activity}
\NormalTok{mod }\OtherTok{=} \FunctionTok{glmer}\NormalTok{(activity\_count }\SpecialCharTok{\textasciitilde{}}\NormalTok{ floor }\SpecialCharTok{+}\NormalTok{ (}\DecValTok{1}\SpecialCharTok{|}\NormalTok{county), }\AttributeTok{data=}\NormalTok{radon, }\AttributeTok{family=}\NormalTok{poisson)}
\FunctionTok{summary}\NormalTok{(mod)}
\end{Highlighting}
\end{Shaded}

\begin{verbatim}
## Generalized linear mixed model fit by maximum likelihood (Laplace
##   Approximation) [glmerMod]
##  Family: poisson  ( log )
## Formula: activity_count ~ floor + (1 | county)
##    Data: radon
## 
##      AIC      BIC   logLik deviance df.resid 
##  24326.1  24340.5 -12160.1  24320.1      875 
## 
## Scaled residuals: 
##    Min     1Q Median     3Q    Max 
## -8.977 -3.176 -1.020  1.778 50.810 
## 
## Random effects:
##  Groups Name        Variance Std.Dev.
##  county (Intercept) 0.2079   0.456   
## Number of obs: 878, groups:  county, 85
## 
## Fixed effects:
##             Estimate Std. Error z value Pr(>|z|)    
## (Intercept)  3.97675    0.05014   79.31   <2e-16 ***
## floor       -0.58897    0.01687  -34.91   <2e-16 ***
## ---
## Signif. codes:  0 '***' 0.001 '**' 0.01 '*' 0.05 '.' 0.1 ' ' 1
## 
## Correlation of Fixed Effects:
##       (Intr)
## floor -0.050
\end{verbatim}

\end{document}
